\HeaderA{race.info}{Provide information on the race}{race.info}
\keyword{misc}{race.info}
\begin{Description}\relax
This function is to be provided by the user. It's
definition has to be given (together with the one of
\code{\LinkA{race.wrapper}{race.wrapper}}) in a file, and the name of such file
has to be passed as first argument to the function \code{race}.
\end{Description}
\begin{Usage}
\begin{verbatim}race.info(data)\end{verbatim}
\end{Usage}
\begin{Arguments}
\begin{ldescription}
\item[\code{data}] It is the object of type \code{list} (possibly empty)
returned by \code{\LinkA{race.init}{race.init}}, if the latter is defined by the
user.
\end{ldescription}
\end{Arguments}
\begin{Value}
The function \code{race.info} is expected
to return a list including the following components:

\begin{ldescription}
\item[\code{race.name}] The name of the race for documentation
purposes.
\item[\code{no.candidates}] The number of candidates in the
race.
\item[\code{no.tasks}] Number of tasks available for testing.
\item[\code{no.subtasks}] Each task might consists of \code{no.subtasks} 
subtasks. If the element \code{no.subtasks} is not included in
the list, it is assumed that each task is indeed atomic,
that is, \code{no.subtasks=1}. \code{no.subtasks} may also be a
vector of length \code{no.tasks}. In this case, the i-th task 
consists of \code{no.subtasks[i]} subtasks.
\item[\code{extra}] A character string providing extra
information on the race for documentation purposes. It can be a long
string and the user is not required to introduce newline characters:
it will be automatically formatted for pretty-printing.
\end{ldescription}
\end{Value}
\begin{Author}\relax
Mauro Birattari
\end{Author}
\begin{SeeAlso}\relax
\code{\LinkA{race}{race}}, \code{\LinkA{race.init}{race.init}}
\end{SeeAlso}
\begin{Examples}
\begin{ExampleCode}
# Please have a look at the function `race.info'
# defined in the file `example-wrapper.R':
local({
  source(file.path(system.file(package="race"),
                           "examples","example-wrapper.R"),local=TRUE);
  print(race.info)})
\end{ExampleCode}
\end{Examples}

