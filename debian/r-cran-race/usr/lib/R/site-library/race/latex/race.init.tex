\HeaderA{race.init}{Initialization function}{race.init}
\keyword{misc}{race.init}
\begin{Description}\relax
This function may be provided by the user for initializing
the computation of the slave processes. It's definition has to be given
in the same file in which the functions \code{race.wrapper} and
\code{race.info} are defined.
The name of such file has to be passed as first argument to the
function \code{race}.
\end{Description}
\begin{Usage}
\begin{verbatim}race.init()\end{verbatim}
\end{Usage}
\begin{Arguments}
The function \code{race.init} has to be called with no
arguments.
\end{Arguments}
\begin{Details}\relax
This function should be used for initializing the computation
on each slave, e.g. loading libraries or data needed by
\code{race.wrapper}, \code{race.info}, and/or
\code{race.describe}. The output of
\code{race.init} will be passed to these functions.
\end{Details}
\begin{Value}
The function \code{race.init} is expected to return an object of
mode list.
\end{Value}
\begin{Author}\relax
Mauro Birattari
\end{Author}
\begin{SeeAlso}\relax
\code{\LinkA{race}{race}}, \code{\LinkA{race.wrapper}{race.wrapper}},
\code{\LinkA{race.info}{race.info}}, \code{\LinkA{race.describe}{race.describe}}
\end{SeeAlso}
\begin{Examples}
\begin{ExampleCode}
# Please have a look at the function `race.init'
# defined in the file `example-wrapper.R':
local({
  source(file.path(system.file(package="race"),
                           "examples","example-wrapper.R"),local=TRUE);
  print(race.init)})
\end{ExampleCode}
\end{Examples}

