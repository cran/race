\HeaderA{race.describe}{Describe a candidate}{race.describe}
\keyword{misc}{race.describe}
\begin{Description}\relax
This function may be provided by the user for giving a
description of a candidate. It's definition has to be given
in the same file in which the functions \code{race.wrapper} and
\code{race.info} are defined.
The name of such file has to be passed as first argument to the
function \code{race}.
\end{Description}
\begin{Usage}
\begin{verbatim}race.describe(candidate,data)\end{verbatim}
\end{Usage}
\begin{Arguments}
\begin{ldescription}
\item[\code{candidate}] The candidate for which a description is to be
returned.
\item[\code{data}] It is the object of type \code{list} (possibly empty)
returned by \code{\LinkA{race.init}{race.init}}, if the latter is defined by the
user.
\end{ldescription}
\end{Arguments}
\begin{Value}
The function \code{race.describe} should return an object
describing the selected candidate. Such object will be printed by
\code{race} through the function \code{print}.
\end{Value}
\begin{Author}\relax
Mauro Birattari
\end{Author}
\begin{SeeAlso}\relax
\code{\LinkA{race}{race}}, \code{\LinkA{race.init}{race.init}}
\end{SeeAlso}
\begin{Examples}
\begin{ExampleCode}
# Please have a look at the function `race.describe'
# defined in the file `example-wrapper.R':
local({
  source(file.path(system.file(package="race"),
                           "examples","example-wrapper.R"),local=TRUE);
  print(race.describe)})
\end{ExampleCode}
\end{Examples}

