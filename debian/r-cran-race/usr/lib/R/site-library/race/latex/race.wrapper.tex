\HeaderA{race.wrapper}{Test a candidate on a task}{race.wrapper}
\keyword{misc}{race.wrapper}
\begin{Description}\relax
This function is to be provided by the user. It's
definition has to be given (together with the one of \code{race.info})
in a file, and the name of such file has to
be passed as first argument to the function \code{race}.
\end{Description}
\begin{Usage}
\begin{verbatim}race.wrapper(candidate,task,data)\end{verbatim}
\end{Usage}
\begin{Arguments}
\begin{ldescription}
\item[\code{candidate}] The candidate to be evaluated: a number between 1 and
\code{no.candidates}, where \code{no.candidates} is the number of
candidates and is to be defined within the function
\code{race.wrapper} itself.
\item[\code{task}] The task on which to the candidate should be evaluated: a
number between 1 and \code{no.tasks}, where \code{no.tasks} is the
number of tasks available for testing, and is to be defined within
the function \code{race.wrapper} itself.
\item[\code{data}] It is the object of type \code{list} (possibly empty)
returned by \code{\LinkA{race.init}{race.init}}, if the latter is defined by the
user.
\end{ldescription}
\end{Arguments}
\begin{Value}
A number: the result obtained by the given candidate at the given
task. If \code{no.subtasks>1} (see \code{\LinkA{race.info}{race.info}}), the
function is expected to return a vector of length equal to
\code{no.subtasks} where the component \code{k} of such vector is the
result obtained by the given candidate on the \code{k}-th subtask
composing the given task.
\end{Value}
\begin{Note}\relax
Please notice that \code{race} is a \bold{minimization} algorithm:
it selects the candidate that obtains the smallest results on
the various tasks considered.
\end{Note}
\begin{Author}\relax
Mauro Birattari
\end{Author}
\begin{SeeAlso}\relax
\code{\LinkA{race}{race}}, \code{\LinkA{race.init}{race.init}},
\code{\LinkA{race.info}{race.info}}
\end{SeeAlso}
\begin{Examples}
\begin{ExampleCode}
# Please have a look at the function `race.wrapper'
# defined in the file `example-wrapper.R':
local({
  source(file.path(system.file(package="race"),
                           "examples","example-wrapper.R"),local=TRUE);
  print(race.wrapper)})
\end{ExampleCode}
\end{Examples}

