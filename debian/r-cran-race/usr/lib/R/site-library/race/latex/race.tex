\HeaderA{race}{Racing methods for the selection of the best}{race}
\keyword{design}{race}
\keyword{htest}{race}
\keyword{optimize}{race}
\begin{Description}\relax
Implementation of some racing methods for the empirical
selection of the best. If the \R{} package \code{rpvm} is installed
(and PVM is available, properly configured, and initialized), the
evaluation of the candidates are performed in parallel on different
hosts.
\end{Description}
\begin{Usage}
\begin{verbatim}
race(wrapper.file, maxExp=0,
        stat.test=c("friedman","t.bonferroni","t.holm","t.none"),
           conf.level=0.95, first.test=5, interactive=TRUE,
              log.file="", no.slaves=0,...)
\end{verbatim}
\end{Usage}
\begin{Arguments}
\begin{ldescription}
\item[\code{wrapper.file}] The name of a file containing the definition of 
the functions to be provided by the user: i.e.
\code{\LinkA{race.wrapper}{race.wrapper}} and \code{\LinkA{race.info}{race.info}}. The file
\code{wrapper.file} might also define the functions
\code{\LinkA{race.init}{race.init}} and \code{\LinkA{race.describe}{race.describe}}.
\item[\code{maxExp}] Maximum number of experiments (i.e. evaluations of the
function \code{race.wrapper}) that are allowed before selecting the
best candidate. If \code{maxExp=0}, no limit is imposed\ldots very
unrealistic in practice.
\item[\code{stat.test}] Statistical test to be used for discarding inferior
candidates.
\item[\code{conf.level}] The confidence level to be used for the statistical
test.
\item[\code{first.test}] The first test for discarding inferior candidates is
performed only when all candidates have been evaluated on a minimum
number of tasks equal to \code{first.test}.
\item[\code{interactive}] If \code{TRUE}, print a progress report on the
standard output.
\item[\code{log.file}] File for saving periodically the state of the race.
\item[\code{no.slaves}] When running under PVM, \code{no.slaves} specify the
number of slaves to be spawned. If \code{no.slave=0} PVM is not used
and all experiments are performed on the local host.
\item[\code{...}] All extra parameters are passed to the function
\code{\LinkA{race.init}{race.init}} defined by the user in the file
\code{wrapper.file}.
\end{ldescription}
\end{Arguments}
\begin{Details}\relax
This package implemets some racing procedures for selecting
from a set of candidate the one that is able to yield the best
performance on a given set of tasks. The time available for selecting
the best candidate is limited and, therefore, a brute-force approach is
unfeasible.  The algorithm implemented in this package sequentially
evaluates the set of candidatas on the available tasks while discards
bad candidates as soon as statistically sufficient evidence is gathered
against them.  The elimination of inferior candidates, speeds up the
procedure and allows a more reliable evaluation of the promising ones.
\end{Details}
\begin{Value}
The output of \code{race} is a list containing the following
components:

\begin{ldescription}
\item[\code{precis}] A string describing the race for documentation
purposes.
\item[\code{results}] A matrix containing in position \code{[i,j]} the result
obtained by candidate \code{j} on task \code{i}.
\item[\code{no.candidates}] Number of candidates at the beginning of the race.
\item[\code{no.tasks}] Number of tasks on which the selection was based.
\item[\code{no.subtasks}] Number of subtasks composing each tasks. Default=1
\item[\code{no.experiments}] Number of times that the function
\code{race.wrapper} had to be call in order to select the best.
\item[\code{no.alive}] Number of candidates that completed the race, that is,
number of candidates that had not been discarded at the moment in
which the race was stopped.
\item[\code{alive}] List of the candidates that completed the race:
no sufficient evidence was gathered, give that the test
\code{stat.test} is adopted, for stating that these candidates are
worse than the selected best.
\item[\code{alive.inTime}] Number of candidates in the race after each time
step.
\item[\code{best}] The candidate selected in the race.
\item[\code{mean.best}] The average result of the best on the tasks
considered.
\item[\code{description.best}] An object describing the selected candidate.
\item[\code{timestamp.start}] Time stamp of the beginning of the race.
\item[\code{timestamp.end}] Time stamp of the end of the race.
\end{ldescription}
\end{Value}
\begin{Note}\relax
Please notice that \code{race} is a \bold{minimization} algorithm:
it selects the candidate that obtains the smallest results on
the various tasks considered.
\end{Note}
\begin{Author}\relax
Mauro Birattari
\end{Author}
\begin{References}\relax
O. Maron and A.W. Moore (1994) Hoeffding Races: Accelerating Model
Selection Search for Classification and Function Approximation.
\emph{Advances in Neural Information Processing Systems 6},
pp. 59--66. Morgan Kaufmann.

A.W. Moore and M.S. Lee (1994) Efficient Algorithms for Minimizing
Cross Validation Error. \emph{International Conference on Machine Learning},
pp. 190--198. Morgan Kaufmann.

O. Maron and A.W. Moore (1997) The Racing Algorithm: Model Selection
for Lazy Learners. \emph{Artificial Intelligence Review},
\bold{11}(1--5), pp. 193--225.

M. Birattari, T. Stuetzle, L. Paquete, and K. Varrentrapp
(2002) A Racing Algorithm for Configuring Metaheuristics. 
\emph{GECCO 2002: Genetic and Evolutionary Computation Conference},
pp. 11--18. Morgan Kaufmann.

M. Birattari (2004) \emph{The Problem of Tuning Metaheuristics 
as Seen from a Machine Learning Perspective}. PhD Thesis, 
Universite' Libre de Bruxelles, Brussels, Belgium.
\end{References}
\begin{SeeAlso}\relax
\code{\LinkA{race.wrapper}{race.wrapper}}, \code{\LinkA{race.info}{race.info}},
\code{\LinkA{race.init}{race.init}}, \code{\LinkA{race.describe}{race.describe}}
\end{SeeAlso}
\begin{Examples}
\begin{ExampleCode}
# The wrapper and init functions for this example are defined in the
# file examples/example-wrapper.R in the installation directory of the
# package.  Please, have a look at such file before implementing your
# own wrapper.
# This example require the package `nnet'
if (require(nnet)&&require(datasets)){
  example.wrapper<-file.path(system.file(package="race"),
                           "examples","example-wrapper.R")
  # Run the race
  race(example.wrapper)

  # If the package `rpvm' is installed on your system and if PVM is
  # properly installed and configured, you can try the following:
  #race(example.wrapper,no.slaves=6)
}
\end{ExampleCode}
\end{Examples}

